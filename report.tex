\documentclass[12pt]{article}

\usepackage{fullpage}
\usepackage{multicol,multirow}
\usepackage{tabularx}
\usepackage{ulem}
\usepackage[utf8]{inputenc}
\usepackage[russian]{babel}
\usepackage{minted}

\usepackage{color} %% это для отображения цвета в коде
\usepackage{listings} %% собственно, это и есть пакет listings

\lstset{ %
language=C,                 % выбор языка для подсветки (здесь это С)
basicstyle=\small\sffamily, % размер и начертание шрифта для подсветки кода
numbers=left,               % где поставить нумерацию строк (слева\справа)
%numberstyle=\tiny,           % размер шрифта для номеров строк
stepnumber=1,                   % размер шага между двумя номерами строк
numbersep=5pt,                % как далеко отстоят номера строк от подсвечиваемого кода
backgroundcolor=\color{white}, % цвет фона подсветки - используем \usepackage{color}
showspaces=false,            % показывать или нет пробелы специальными отступами
showstringspaces=false,      % показывать или нет пробелы в строках
showtabs=false,             % показывать или нет табуляцию в строках
frame=single,              % рисовать рамку вокруг кода
tabsize=2,                 % размер табуляции по умолчанию равен 2 пробелам
captionpos=t,              % позиция заголовка вверху [t] или внизу [b] 
breaklines=true,           % автоматически переносить строки (да\нет)
breakatwhitespace=false, % переносить строки только если есть пробел
escapeinside={\%*}{*)}   % если нужно добавить комментарии в коде
}


\begin{document}
\begin{titlepage}
\begin{center}
\textbf{МИНИСТЕРСТВО ОБРАЗОВАНИЯ И НАУКИ РОССИЙСОЙ ФЕДЕРАЦИИ
\medskip
МОСКОВСКИЙ АВЦИАЦИОННЫЙ ИНСТИТУТ
(НАЦИОНАЛЬНЫЙ ИССЛЕДОВАТЬЕЛЬСКИЙ УНИВЕРСТИТЕТ)
\vfill\vfill
{\Huge ЛАБОРАТОРНАЯ РАБОТА №5} \\
по курсу объектно-ориентированное программирование
I семестр, 2019/20 уч. год}
\end{center}
\vfill

Студент \uline{\it {Попов Данила Андреевич, группа М8О-208Б-18}\hfill}

Преподаватель \uline{\it {Журавлёв Андрей Андреевич}\hfill}

\vfill
\end{titlepage}

\subsection*{Условие}

Работа с контейнерами

Реализовать структуру данных ОЧЕРЕДЬ.

\subsection*{Дневник отладки}

Очень ленивая имплементация CLI. 

\subsection*{Недочёты}

Аллокатор не копируется, работает только с единичными объектами.

\subsection*{Выводы}

Ха, 1 курс с C++! Научился НЕ копировать unique\_ptr, перемещать указатели через release и reset, использовать нестандратные аллокаторы с интерфейсом STL. При первых попытках реализовать вектор познакомился с compressed\_pair, и мне не понятно, почему данной структуры нет в стандарте. Лаборатнорная понравилась, много полезного опыта, несмотря на то, что он не попал в конечный вариант решения лабораторной.

\vfill

\subsection*{Исходный код}

{\Huge main.cpp}
\inputminted
    {C++}{app/main.cpp}
    \pagebreak

{\Huge include/point.hpp}
\inputminted
    {C++}{include/point.hpp}
    \pagebreak

{\Huge include/polygon.hpp}
\inputminted
    {C++}{include/polygon.hpp}
    \pagebreak
    
\end{document}
